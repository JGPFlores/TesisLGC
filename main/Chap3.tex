% **************************************************
\chapter{Título del capítulo 3: Versión laraga del título}
\chaptermark{Versión corta del título}
% **************************************************
\chapterquote{Escribe una frase aquí para iniciar el \textsc{Capítulo 3}\ldots}{Autor de la frase}{Datos adicionales aquí, \oldstylenums{2004}}
% **************************************************


% Resumen:
\begin{ResumenPhDThesis}
\begin{changemargin}{1cm}{1cm}
\lettrine[lines=3]{E}{scribir aquí} el resumen del \textsc{Capítulo 3}: \lipsum[3]
\end{changemargin}
\end{ResumenPhDThesis}


% **************************************************
\section{Introducción}
% **************************************************
Escribir en este apartado, una introducción del \textsc{Capítulo 3}: \lipsum[1]


% **************************************************
\section{Uso de ecuaciones}
% **************************************************
La Ecuación~\ref{Eq:BoxBehnken} es una ecuación polinomial que se describe a continuación \citep{du2014beta}.

\begin{equation}\label{Eq:BoxBehnken}
Y = \beta_{0} + \sum_{i=1}^{k} \beta_{i} X_{i} + \sum_{i=1}^{k} \beta_{ii} X_{i}^{2} + \sum_{i=1}^{k-1} \sum_{j=i+1}^{k} \beta_{ij} X_{i} X_{j}
\end{equation}


Donde:
\begin{description}
	\item $Y$ es la respuesta predicha
	\item $\beta_{0}$ es el término compensatorio
	\item $\beta_{i}$ es el coeficiente del efecto lineal ($X_{i}$)
	\item $\beta_{ii}$ es el coeficiente del efecto cuadrático ($X_{i}^{2}$)
	\item $\beta_{ij}$ es el coeficiente del efecto de interacción lineal-lineal ($X_{i} X_{j}$)
\end{description}

Texto adicional: \lipsum[5]


% **************************************************
\section{Conclusiones}
% **************************************************
Escribir las conclusiones del \textsc{Capítulo 3}: \lipsum[2]


% **************************************************
% Referencias
% **************************************************
\begin{refcontext}[sorting=nyt]
\printbibliography[title={Referencias}, heading=subbibintoc]
\end{refcontext}
