% **************************************************
\chapter{Título del capítulo 2: Versión laraga del título}
\chaptermark{Versión corta del título}
% **************************************************
\chapterquote{Escribe una frase aquí para iniciar el \textsc{Capítulo 2}\ldots}{Autor de la frase}{Datos adicionales aquí, \oldstylenums{2009}}
% **************************************************

% Resumen:
\begin{ResumenPhDThesis}
\begin{changemargin}{1cm}{1cm}
\lettrine[lines=3]{E}{scribir aquí} el resumen del \textsc{Capítulo 2}: \lipsum[3]
\end{changemargin}
\end{ResumenPhDThesis}



% **************************************************
\section{Introducción}
% **************************************************
Escribir en este apartado, una introducción del \textsc{Capítulo 2}: \lipsum[1]

Uso de listas con números romanos: 

\begin{enumerate}[label=\roman*.]
	\item Paso uno;
	\item Paso dos;
	\item Paso tres;
	\item Paso cuatro;
	\item Paso cinco.
\end{enumerate}


% **************************************************
\section{Escribe el nombre de la sección del capítulo}
\index{niveles!sección del capítulo}
% **************************************************
\lipsum[2]

% **************************************************
\subsection{Escribe el nombre de la subsección}
\index{niveles!subsección del capítulo}
% **************************************************
\lipsum[2]

% **************************************************
\subsubsection{Escribe el nombre de la subsubsección}
\index{niveles!subsubsección del capítulo}
% **************************************************
\lipsum[2]

% **************************************************
\paragraph{Escribe un párrafo} % Esto es en vez de una sub-sub-subsección 
\index{niveles!párrafo}
% **************************************************
\lipsum[2]



% **************************************************
\section{Figuras}
% **************************************************
\subsection{Una figura}
\index{insertar!figura}
% **************************************************
A continuación se presenta el ejemplo de cómo insertar la Figura~\ref{fig:EjemploFigura}:

\begin{figure}[htp]
	\centering
	\includegraphics[width=1\linewidth]{FiguraX.png}
    \caption[Título de una figura.]% Así es como aparece en el índice de figuras
    {Título de una figura con una descripción más detallada. \lipsum[1]} % Se puede redactar una descripción con más detalles
    \label{fig:EjemploFigura}
\end{figure}


% **************************************************
\subsection{Una figura con rotación}
\index{insertar!figura con rotación}
% **************************************************

En la Figura~\ref{fig:FiguraRotacion} se muestra una figura con rotación:

%Insertar una figura con rotación
\begin{sidewaysfigure}[htp]
	\centering 
	\includegraphics[width=1\linewidth]{FiguraX.png}
    \caption[Título de una figura con rotación.]% Título que aparece en el índice de figuras
    {Título de una figura con rotación con una descripción más detallada. \lipsum[1]} % Descripción más larga de la figura
    \label{fig:FiguraRotacion}
\end{sidewaysfigure}


% **************************************************
\subsection{Subfigura}
\index{insertar!subfiguras}
% **************************************************
En la Figura~\ref{Fig:SubfigurasEjemplo} se incluyen subfiguras, es decir: La Figura~\ref{Fig:SubFig1} y la Figura~\ref{Fig:SubFig2}. 


\begin{figure}[htp]
   \centering
   %%---- Primera subfigura ----
   \subfloat[Título de la subfigura 1]{ %Es opcional
        \label{Fig:SubFig1}         %% Etiqueta para la primera subfigura
        \includegraphics[width=0.455\textwidth]{FiguraX.png}}
   \hspace{0.05\linewidth}
   %%---- Segunda subfigura ----
   \subfloat[Título de la subfigura 2]{ %Es opcional
        \label{Fig:SubFig2}         %% Etiqueta para la segunda subfigura
        \includegraphics[width=0.455\textwidth]{FiguraX.png}}
   \caption[Título de una figura con subfiguras.]%
    {Título de una figura con subfiguras con una descripción más detallada. \lipsum[5]}
   \label{Fig:SubfigurasEjemplo} % Etiqueta para la figura entera
\end{figure}


% **************************************************
\subsection{Subfiguras con rotación}
\index{insertar!subfiguras con rotación}
% **************************************************
La Figura~\ref{Fig:RotacionSubfiguras} muestra la rotación de subfiguras. Y en el texto se deben citar las subfiguras que componen esta figura: Figura~\ref{Fig:subfigA}, Figura~\ref{Fig:subfigB} y Figura~\ref{Fig:subfigC}.


\begin{sidewaysfigure}[htp]
   \centering
   %%---- Primera subfigura ----
   \subfloat[Descripción de la subfigura 1]{
        \label{Fig:subfigA}         %% Etiqueta para la primera subfigura
        \includegraphics[width=0.45\textwidth]{FiguraX.png}}
   \hspace{0.05\linewidth}
     %%---- Segunda subfigura ----
   \subfloat[Descripción de la subfigura 2]{
        \label{Fig:subfigB}         %% Etiqueta para la segunda subfigura
        \includegraphics[width=0.45\textwidth]{FiguraX.png}}
   \hspace{0.05\linewidth}
   %%---- Tercera subfigura ----
   \subfloat[Descripción de la subfigura 3]{
        \label{Fig:subfigC}         %% Etiqueta para la tercer subfigura
        \includegraphics[width=0.45\textwidth]{FiguraX.png}}
   \caption[Título de una figura con subfiguras con rotación.]% Título de la figura como aparece en el índice de figuras 
    {Título de naa figura con subfiguras con rotación con una descripción más detallada. \lipsum[5]} % Título de la figura con una descripción detallada
   \label{Fig:RotacionSubfiguras} % Etiqueta para la figura entera
\end{sidewaysfigure}



% **************************************************
\subsection{Subfiguras en diferentes páginas}
\index{insertar!subfiguras en diferentes páginas}
% **************************************************

La Figura~\ref{fig:SubfigurasMultiplesPaginas} es un ejemplo de subfiguras en diferentes páginas. Adicionalmente, en el texto se citan cada una de las subfiguras: Figura~\ref{fig:SubfiguraX}, Figura~\ref{fig:SubfiguraY} y Figura~\ref{fig:SubfiguraZ}.


%--Dividir un subfiguras en múltiples páginas
\newpage
\begin{figure}[thbp]
	\centering
	\subfloat[Título de la primer subfigura.]{
	\label{fig:SubfiguraX} %% Etiqueta para la primer subfigura
	\includegraphics[width=1\textwidth]{FiguraX.png}}
	\caption[Título de una figura completa con subfiguras en diferentes páginas.]{Título de una figura completa con subfiguras en diferentes páginas.}
\label{fig:SubfigurasMultiplesPaginas} % Etiqueta para la figura entera
\end{figure}


\newpage
\begin{figure}[thbp]
\ContinuedFloat
	\centering
	\subfloat[Título de la segunda subfigura.]{
	\label{fig:SubfiguraY} %% Etiqueta para la segunda subfigura
	\includegraphics[width=1\textwidth]{FiguraX.png}}
	\caption[]{Título de una figura completa con subfiguras en diferentes páginas (cont.).} %Continuación
\end{figure}


\newpage
\begin{figure}[thbp]
\ContinuedFloat
	\centering
	\subfloat[Título de la tercer subfigura.]{
	\label{fig:SubfiguraZ} %% Etiqueta para la tercer subfigura
	\includegraphics[width=1\textwidth]{FiguraX.png}}
	\caption[]{Título de una figura completa con subfiguras en diferentes páginas (cont.).} %Última parte.
\end{figure}



% **************************************************
\newpage
\section{Tablas}
% **************************************************
\subsection{Tabla}
\index{insertar!tabla}
% **************************************************
A continuación se presenta el ejemplo de cómo insertar la Tabla~\ref{Tab:EjemploTabla}, que además contiene notas al pie de tabla:


\begin{table}[htbp]
	\caption{Título de una tabla.}
	\begin{center}
	\begin{tabular}{ l l }
	\toprule
	\parnoteclear
	\parnote{Resultados expresados en g 100g\textsuperscript{-1} (bs).}Componente &
	Contenido \\ 		
	\midrule
		Caramelo & 567$\pm$0.28 \\
		Gomitas	& 945$\pm$0.22 \\
		Paletas & 736$\pm$0.20 \\
		Marshmallows & 978$\pm$0.34 \\
		Chocolates & 527$\pm$0.36 \\
		\parnote{Expresado en otras unidades.}Sustancia X	& 875$\pm$0.56 \\
	\bottomrule
\end{tabular}
\end{center}
\label{Tab:EjemploTabla}
\parnotes
\end{table}


% **************************************************
\subsection{Tabla con rotación}
\index{insertar!tabla con rotación}
% **************************************************
La Tabla~\ref{tab:TablaRotacion} es un ejemplo de una tabla con rotación y con notas al pie de tabla.


\begin{sidewaystable}[htbp]
\caption{Título de una tabla con rotación.}
\centering
\begin{tabular}{ c c c c c c c c c c c }
\toprule %------
\parnoteclear
	Exp. 
	& \parnote{$X_{1}$: concentración molar, mol L\textsuperscript{-1}.}$X_{1}$  
	& \parnote{$X_{2}$: tiempo, h.}$X_{2}$  
	& \parnote{$X_{3}$: temperatura, $\celsius$.}$X_{3}$  
	& \parnote{$Y_{1}$: rendimiento, \% p/p (bs).}$Y_{1}$  
	& \parnote{$Y_{2}$: masa, g.}$Y_{2}$  
	& \parnote{$Y_{3}$: viscosidad intrínseca ([$\eta$]), mL g\textsuperscript{-1}.}$Y_{3}$  
	& \parnote{$Y_{4}$: masa molar (MW), kDa.}$Y_{4}$ 
	& \parnote{$Y_{5}$: relación arabinosa-xilosa (Ara/Xyl).}$Y_{5}$ 
	& \parnote{$Y_{6}$: contenido de ácidos hidroxicinámicos, $\mu$g mg\textsuperscript{-1} de AX.}$Y_{6}$ 
	& \parnote{$Y_{7}$: potencial zeta ($\zeta$), mV.}$Y_{7}$ \\ 
\midrule %------
	1 & 0.1 & 10 & 60 & 1$\pm$0.1 & 6.73$\pm$1.02 & 50.37$\pm$0.87 & 45.85$\pm$0.45 & 0.30$\pm$0.01 & 9.47$\pm$0.12 & -16.18$\pm$1.13 \\
\bottomrule %------
\end{tabular}
\label{tab:TablaRotacion}
\begin{flushleft}
\parnotes
\end{flushleft}
\end{sidewaystable}


% **************************************************
\newpage
\subsection{Tabla extensa en varias páginas}
\index{insertar!tabla extensa}
% **************************************************
La Tabla~\ref{tab:LongTab1} es un ejemplo de tabla que puede extenderse por varias páginas si se requiere. 

%Las tablas grandes, se pueden incluir en archivos TeX independientes para un manejo más cómodo del código
\subfile{main/LongTable1}



% **************************************************
\section{Otras citas}
% **************************************************
Se pueden utilizar los comandos de Bib{\TeX} para citar:

\citep{iqbal2011evaluation, kamboj2014physicochemical, malunga2017effect, moate2011influence}

\citet{dima2016kinetics}

Por ejemplo:

\citet{ahmadi2012development, zhou2014validation, bagchi2016studies} reportaron que\ldots

Los resultados son acordes a los reportados en otras investigaciones \citep{vsimkovic2011positively, hromadkova2013structural, xiong2013antioxidant, coelho2016revisiting}\ldots


% **************************************************
\section{Conclusiones}
% **************************************************
Escribir las conclusiones del \textsc{Capítulo 2}: \lipsum[2]

% **************************************************
% Referencias
% **************************************************
\begin{refcontext}[sorting=nyt]
\printbibliography[title={Referencias}, heading=subbibintoc]
\end{refcontext}
