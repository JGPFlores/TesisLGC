% **************************************************
%	Configuración del idioma:
% **************************************************
\usepackage[utf8]{inputenc}
\usepackage[english, spanish, es-tabla]{babel}
\selectlanguage{spanish}


% **************************************************
%	Paquetes:
% **************************************************
\usepackage{lipsum} % Incluir texto en latín de relleno
\usepackage{subfiles} %incluir tablas a manera de sub-archivos
\usepackage[version=4]{mhchem} %escribir símbolos químicos en el texto
\usepackage{chemfig} %estructuras químicas
\usepackage{amsmath,amsthm,amsfonts,amssymb,gensymb,textcomp}
\usepackage{fourier-orns}
\usepackage{graphicx} % Insertar figuras
\usepackage{subfig} % Insertar subfiguras
\usepackage{lscape} %no rotation of the page 
\usepackage{listings}
\usepackage{tikz}
\usepackage[final]{pdfpages} %Configuración de PDF Externos
\usepackage{makeidx} % Índice alfabético
\usepackage[printonlyused,withpage]{acronym} % Crear lista de acrónimos.
\usepackage{comment} %Insertar entornos de comentarios
\usepackage{enumitem} %Enlistar con números romanos
\usepackage[numbib]{tocbibind} %Agrega índices referencias y tablas de contenidos
\spanishdecimal{.} %Cambia las comas por puntos en los números


% **************************************************
%	Configuración de secciones:
% **************************************************
\usepackage{titlesec}
\setcounter{secnumdepth}{3} %enumerar sub-subsecciones
%\setcounter{tocdepth}{3}  %para que las sub-subsecciones aparezcan en el índice


% **************************************************
%	Formato de hipervínculos y datos del editor:
% **************************************************
\usepackage{hyperref}
\hypersetup{
	colorlinks=true,
	citecolor=iColor003,
	filecolor=iColor003,
	linkcolor=iColor003,
	urlcolor=iColor003,
	bookmarksopen=true,
	pdfpagelayout=TwoPageRight,
	pdftitle={Título de la tesis},
	pdfauthor={Jesús Guadalupe Pérez Flores},
	pdfsubject={Tesis doctoral},
	pdfproducer={LaTeX},
	pdfkeywords={Palabras, clave, de, la, tesis}
}


% **************************************************
%	Ajustar longitud de la barra sobre la X:
% **************************************************
\makeatletter
\newcommand*{\Xbar}{}%
\DeclareRobustCommand*{\Xbar}{%
  \mathpalette\@Xbar{}%
}
\newcommand*{\@Xbar}[2]{%
  \sbox0{$#1\mathrm{x}\m@th$}%
  \sbox2{$#1x\m@th$}%
  \rlap{%
    \hbox to\wd2{%
      \hfill
      $\overline{%
        \vrule width 0pt height\ht0 %
        \kern\wd0 %
      }$%
    }%
  }%
  \copy2 %
}
\makeatother
