%\begingroup
\clearpage
\begin{changemargin}{2cm}{2cm} 

    \thispagestyle{empty}
    \vspace*{\fill}
    
%    \dedicatoriafont

\begin{center}
    Dedico esta \textsc{Tesis}:
\end{center}

\begin{center}
    A~\textsc{mi madre}\ldots
\end{center}

\begin{center}
    A~\textsc{mi hermana}\ldots
\end{center}

\begin{center}
    A~\textsc{Laura García Curiel}, eres liberosis en un mundo que en general es bastante hostil, pero que vale la pena vivirlo. Eres paz y sosiego en medio de la guerra, y eres primavera a la mitad del invierno. Gracias por amarme, por impulsarme y por creer en mí\ldots
\end{center}
  
\begin{center}
    \textsc{A todos} los que no respetan los esquemas preestablecidos, a los soñadores, a los locos, a los raros y a los incomprendidos, a los inclaudicables, a los que no encajan, a los entusiastas, a los idealistas, a los emprendedores y a los visionarios del mundo. Porque los cambios más importantes para la humanidad, los grandes avances y los grandes descubrimientos los hacen \textsc{ustedes}\ldots
\end{center}
    
\begin{center}
	\usefont{T1}{LobsterTwo-LF}{bx}{it}
	Jesús Guadalupe Pérez Flores
\end{center}

\begin{center}
	\Huge
	\textxswup
\end{center}

    \vspace*{\fill}

\end{changemargin}
\clearpage
%\endgroup
